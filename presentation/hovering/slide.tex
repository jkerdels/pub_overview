


\begin{frame}{Visual Hovering}


\twocol{0.4}{
\vspace{-4em}
\begin{figure}
\adjincludegraphics[width=0.9\linewidth,valign=t]{hovering/schematic.jpg}
\vspace{-0.5em}
\caption{\scriptsize Hovering control scheme with 5 PID controllers~\cite{Kerdels2008a}.}
\adjincludegraphics[width=1.\linewidth,valign=b]{hovering/lbv150.jpg}
\vspace{-1.5em}
\caption{\scriptsize LBV hovering in laboratory test tank 
{\em locked} on the tip of a 3D gantry crane (left). Tracking of visual 
features (right)~\cite{Kerdels2008a}.}
\end{figure}
}{0.55}{
\justifying

\vspace{1em}
In~\cite{Kerdels2008a} we present a {\bf vision-based control} algorithm that 
enables underwater ROV to hover in front of visible structures 
compensating, e.g., drift.

\vspace{2em}
We introduce a {\bf novel approach to automatically tune the used keypoint 
detector} to the level of contrast present in each local region of the camera 
image. This automatic adjustment enables robust, paramter free operation.

\vspace{2em}
The algorithm was successfully tested on a LBV150 ROV 
\vid{publications/2008-03/ROV_LBV150.mp4} by Seabotix both under laboratory 
conditions and in open waters.

}

\vspace{-1em}

\begin{center}
\rule{2cm}{0.4pt}\\[0.5em]
\end{center}

\fc{Kerdels2008a}{publications/2008-03/2008-03}

\end{frame}
