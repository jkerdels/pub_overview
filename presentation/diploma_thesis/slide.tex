


\begin{frame}{Diploma Thesis}

%\vspace{1em}
\justifying
My diploma thesis~\cite{Kerdels2006} presents a novel approach to {\bf discover 
objects in unlabeled image data} using a combination of traditional methods 
including image segmentation, feature extraction, clustering, and dynamic 
programming.

\vspace{1.5em}
The key idea consists of using {\bf image segmentation to group features} in an 
image, and use these feature groups to represent the individual segments in a 
way that is invariant to rotation, scale, and translation.

\vspace{1.5em}
Such feature segments can then be related to each other by an appropriate 
distance measure to {\bf identify segments that occur repeatedly} in different 
contexts.

\vspace{1.5em}
Finally, neighborhood relations among segments can be learned in a similar 
fashion to {\bf discover stable feature segment constellations} that indicate 
the presence of reoccuring structures, i.e., putative objects in the images.

\begin{center}
\rule{2cm}{0.4pt}\\[0.5em]
\end{center}

\fc{Kerdels2006}{publications/2006-01/2006-01}

\end{frame}
